\documentclass[12pt]{article}
\usepackage{amssymb}
\usepackage{amsmath}
\usepackage{bm} % bold maths


\usepackage{fancyhdr}%to have a header and footer
%\pagestyle{fancy}

%\usepackage{longtable} % to split tables over the page

\usepackage{setspace}%this stops lists having so much gap in them
\usepackage{enumitem}
\setlist{nolistsep}


% for xfig
\usepackage{graphicx}
\usepackage[usenames,dvipsnames]{color}
%\usepackage[pdftex]{graphicx}
%\DeclareGraphicsExtensions{.pdf, .jpg}
\usepackage{epsfig}
% end for xfig

\usepackage[mathscr]{eucal}%Euler script, use \mathscr


% external references
%\usepackage{xr}
%\externaldocument[TS-]{tgt-splash-paper}

\setlength{\topmargin}{-1.5cm} 
\setlength{\textheight}{22cm}
\setlength{\oddsidemargin}{-0.2cm}
\setlength{\evensidemargin}{-1.2cm} 
\setlength{\textwidth}{16cm}
\setlength{\parindent}{0pt}
\setlength{\parskip}{.35cm}


\newtheorem{theorem}{Theorem}[section]
\newtheorem{lemma}[theorem]{Lemma}
\newtheorem{corollary}[theorem]{Corollary}
\newtheorem{defn}[theorem]{Definition}
\newtheorem{definition}[theorem]{Definition}
\newtheorem{example}[theorem]{Example}
\newtheorem{remark}[theorem]{Remark}
\newtheorem{construction}{Construction}
\newtheorem{conjecture}[theorem]{Conjecture}
\newtheorem{result}[theorem]{Result}
\newenvironment{proof}{\noindent{\bf Proof}\hspace{0.5em}}
    { \null  \hfill $\square$ \par}



\newcommand{\X}{\mathcal X}
\newcommand{\Y}{\mathcal Y}
\newcommand{\R}{\mathcal R}
\newcommand\C{{\cal C}}
\newcommand\E{{\cal E}}
\newcommand\D{{\cal D}}
\newcommand\F{{\cal F}}
\newcommand\N{{\cal N}}
\newcommand\J{{\cal J}}
\renewcommand\L{{\mathscr L}}
\renewcommand{\S}{\mathcal S}
\renewcommand{\P}{\mathcal P}
\newcommand{\K}{\mathcal K}
\newcommand{\KK}{\mathscr K}
\renewcommand{\H}{\mathcal H}
\newcommand{\Q}{\mathscr Q}



\newcommand{\takeaway}{\backslash}
\renewcommand\setminus{\backslash}
\newcommand{\st}{:}

\newcommand\PGammaL{{\mbox{P}\Gamma {L}}}
\newcommand\PGL{{\rm PGL}}
\newcommand\GF{{\rm GF}}
\newcommand\PG{{\rm PG}}
\newcommand\AG{{\rm AG}}
\newcommand\PGO{{\rm PGO}}

%\usepackage{fourier}%lets you use \grimace, and changes font

%% to number each row in a tabular environment
%\usepackage{array} 
%\newcounter{exno}
%\newenvironment{table-label-row}
%{
%\begin{flushleft}
%%\hspace*{1cm}
%\begin{tabular}{|>{(\refstepcounter{exno}\theexno)}c|c|c|c|c|}
%}
%{
%\end{tabular}
%\end{flushleft}
%}
%
%


%%%%
\newcommand{\Label}{\label}
\newcommand{\Labele}{\label}
%\newcommand{\LabelTable}[1]{\label{#1}{\rm [Table called {\it #1}]}}
%\newcommand{\Label}[1]{{\blue{\label{#1}\marginpar{\tiny{#1}}}}}
%\newcommand{\Labele}[1]{\label{#1}{\mbox{\quad [{\tt #1}]}}}
%\newcommand\LABLE[1]{{\tiny\color{green} #1}}


%\usepackage{color}%already added above
\newcommand\wenai[1]{{\color{blue} #1}}
\newcommand\sue[1]{{\color{magenta} #1}}
\newcommand\TODO[1]{{\color{Purple} #1}}
\newcommand{\todo}{{\color{red} \ \ ********* TO DO *********\\ } }
\newcommand\red[1]{{\color{red} #1}}
\newcommand\blue[1]{{\color{blue} #1}}
\newcommand\magenta[1]{{\color{magenta} #1}}
\newcommand\green[1]{{\color{green} #1}}
%\newcommand\green[1]{{                 }}


%\renewcommand\baselinestretch{1.5}     
\newcommand\hl{{\;\rule[.5ex]{0.8em}{0.4pt}\;}} 

%srg shortcuts
\newcommand{\clique}{\mathcal C}
\newcommand{\setX}{{\mathcal C}_{{\rm\romannumeral 1}}}
\newcommand{\setY}{{\mathcal C}_{{\rm\romannumeral 2}}}
\newcommand{\setZ}{{\mathcal C}_{{\rm\romannumeral 3}}}
\newcommand\tone{{\rm (\romannumeral 1)}}
\newcommand\ttwo{{\rm (\romannumeral 2)}}
\newcommand\tthree{{\rm (\romannumeral 3)}}
\newcommand\tonen{{\rm  \romannumeral 1}}
\newcommand\ttwon{{\rm  \romannumeral 2}}
\newcommand\tthreen{{\rm  \romannumeral 3}}

\begin{document}
%%Fancy header usage
%\pagestyle{fancy}
%\fancyhf{}
%%%\fancyheadR{} % predefined ()
%%%\fancyhead[L]{\color{Magenta}\leftmark} % 1. subsectionname
%\fancyfoot[C]{\color{Magenta}\thepage}
%\lfoot{\color{Magenta}\jobname}
%\rfoot{\color{Magenta}\today}
%
%\fancypagestyle{plain}{%
%  \fancyhf{}%
%  \renewcommand{\headrulewidth}{0pt}%
%}
%%{\LARGE{\color{green}
%%A new class of strongly regular graphs}
%%}
%
%%{\color{green}
%%\begin{spacing}{0} \tableofcontents \end{spacing}
%%}

\title{Evil Nature Paper :-(}

\author{Shaun Barker, Alex Jackson, Jessica Parker}
%%\date{\today}
%%%\\ Department of Pure Mathematics, University of Adelaide\\
%%%Adelaide 5005, Australia
%%%\\ \\
%}
%%
%%
%%
\maketitle
%
%
%
%%Corresponding Author: Dr Susan Barwick, University of Adelaide, Adelaide
%%5005, Australia. Phone: +61 8 8303 3983, Fax: +61 8 8303 3696, email:
%%susan.barwick@adelaide.edu.au
%
%%Keywords: Bruck-Bose representation, subplanes, exterior splash
%%AMS code: 51E20
%
%
%\newpage
\begin{center}
  \includegraphics[width=.6\linewidth]{i-have-no-idea-what-i-am-doing.jpeg}
  \end{center}

\begin{itemize}
\item Start with 2 lots of DNA \textbf{(alleles?)}
\item Divide sequence into 100 bp bins.
\item Assign each bin either missing (.), heterozygous (1) or homozygous (0). The sequence of this gives the input to the HMM e.g. $(0,1,1,0,0,0,0,0,0,1,0,1,0,\dots)$.
\item Divide time into $0=t_1<t_2<\dots<t_n=T_{\text{max}}<t_{n+1}=\infty$. Choose $T_{\text{max}}$ such that only a few percent of coalescent events fall in $[T_{\text{max}},\infty)$. \textbf{How do you choose this?}
\item Hidden states: which interval $k=[t_k,t_{k+1})$, the TMRCA/coalescent times for the different bins fall into. \textbf{Figure 1} on the paper is quite useful for illustrating this.
\item METHOD...
\item Assume constant population size.
\item Observations into HMM, find TMRCAs of the bins and associated parameters e.g. hidden states, 
\item Evaluate the likelihood of the sequence using EM (expectation maximisation).
\item Powell's direction set: Got a discretised $\lambda(t)$ function (piecewise constant), optimise one ``step''  at a time. \textbf{???}
\item Iterate 20 times.
\end{itemize}


\end{document}
